\subsection{Controller Subsystem Testing}
\label{sec:controller_subsystem_testing}
The MCU subsystem will undergo a littany of testing. The electrical and
logical characteristics, as well as software behavior will be tested to ensure
a working product.

% Electrical characteristics (to see what it can handle and what it uses)
\subsubsection{Electrical Characteristics}
The MCU subsystem should adhere to the following electrical characteristics
related to power draw and voltage input in the following states. Unless noted,
all voltages will be in direct current, and all conditions will be tested at
room temperature (77 $\degree$F). Any use of "system" or "the system" refers
exclusively to, and the entirety of the MCU subsystem.

\paragraph{General Requirements} The system should be able to accept a voltage
of -0.5 to 3.8 V into V\textsubscript{BAT}, -0.5 to 4.3 V into any digital
IO-configured pin, -0.5 to 2.1 V into any RF-configured pin, and -0.5 to 2.1 V
into any analog-configured pin without damage. The system should be able to
accept a current of 0 to 620 mA into any pin without damage. The system should
be able to regulate voltage going into V\textsubscript{BAT}, and provide 3.3 V
(between 3.0 and 3.6 V) on its 3V3 rail to any components in the sensing
subsystem that may require it.

\paragraph{Shutdown Power Draw} The system should not draw more than 10
$\upmu$A in shut down mode. The system should not consume more than 100
$\upmu$W in shut down mode.

\paragraph{Critical Power Draw} The system should not draw more than 1 mA in
critical power mode. The system should not consume more than 5 mW in critical
power mode.

\paragraph{Low Power Draw} The system should not draw more than 100 mA while
receiving or idling in low power mode, and should not draw more than 250
mA while transmitting in low power mode. The system should not consume more
than 500 mW while receiving or idling in low power mode, and should not
consume more than 1.25 W while transmitting in low power mode. The system
should be able to sufficiently power any components in the sensing subsystem in
low power mode.

\paragraph{Active Power Draw} The system should not draw more than 150 mA while
receiving or idling in active mode, and should not draw more than 300
mA while transmitting in active mode. The system should not consume more
than 750 mW while receiving or idling in active mode, and should not
consume more than 1.50 W while transmitting in active mode. The system
should be able to sufficiently power any components in the sensing subsystem in
active mode.

\paragraph{Startup Power Draw} The system should not draw more than 750 mA
within the first 20 seconds of powering on or starting up the system. The
system should not consume more than 4.00 W within the first 20 seconds of
powering on or starting up the system.

\paragraph{Solar Panels and Battery System} The system should be able to
operate nominally solely powered off of the solar and battery subsystem
detailed in \autoref{sec:power_subsystem}. The system should not require any
external power to operate in any modes (e.g. startup, active, etc.).

\paragraph{Operating Temperature} The system should be able to operate
nominally within the temperature range of 30 to 140 $\degree$F. Crystal
oscillators shall fall within the thermal drift ratings provided in the
CC3220 technical reference material.

% Logical characteristics (ability to set and use peripherals n stuff)
\subsubsection{Logical Characteristics}
\paragraph{Upload Programming}
\paragraph{I2C}
\paragraph{Onboard LED}
\paragraph{Watchdog Timer}
\paragraph{Real-time Clock}
\paragraph{WiFi Module}
\paragraph{GPIO}
\paragraph{Interrupts/ISRs}
\paragraph{ADC}
\paragraph{Reset Button}
\paragraph{Startup}

% Software behavior (actual program)
\subsubsection{Software Behavior}
\paragraph{Startup}
\paragraph{Shutdown}
\paragraph{Reset}
\paragraph{Networking}
\paragraph{Internet}
\paragraph{AWS}
\paragraph{Receive (Sockets)}
\paragraph{Send (Sockets)}
\paragraph{Receive (Command)}
\paragraph{Send (Data)}
\paragraph{ISR}
