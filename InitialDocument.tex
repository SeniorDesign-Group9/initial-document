\documentclass[12pt]{article}
\usepackage[top=1in,left=1.5in,right=1in]{geometry}
\usepackage{graphicx}
\usepackage{float}
\usepackage{tabularx}
\usepackage{hyperref}
\hypersetup{
    colorlinks=true,
    linkcolor=blue,
    filecolor=blue,      
    urlcolor=blue}
\begin{document}
\begin{titlepage}
    \begin{center}
        \section*{Auto Garden Bed - Group 9}
        \begin{Large}
            Team members:\\
            \vspace{.5cm}
        \end{Large}
        \begin{large}
            Nicholas Chitty \\
            Brendan College \\
            Scott Peirce \\
            Justin Pham-Trinh \\
        \end{large}
        \vfill
        UCF Senior Design Fall 2022 - Spring 2023

        \pagebreak
    \end{center}
\end{titlepage}
\pagenumbering{gobble}
\tableofcontents
\listoffigures
\listoftables
\pagebreak
\pagenumbering{arabic}
%\input{your tex file here}
\section{Executive Summary}                     % Section 1
New gardeners typically struggle getting their garden started due to a lack of tending to their plants. This project seeks to solve  many of the problems that new gardeners have through sensing and control. The main issues with plant growth relate to soil composition, soil moisture, temperatue, and sun light. This project seeks to use optics to measure the soil moisture and composition; then an MCU will capture this data and control solar shades to control sunlight and solenoids to control watering. A web component will be included to check the weather as well as notify the user of impending weather events that could affect their plants adversively (frost or heat wave). The entire system will be powered with solar panels that are capable of tracking the sun through the sky and can act as blinds over the plants.

This project all starts with scoping out the project. The team has immediately compiling a list of must-have requirements and some things the team would like to accomplish as ``nice to haves". The team started this process by looking at all the similar projects that have already been done and looked at all the ways the team can expand on the work they have already accomplished. For example, the team liked the weather aspects of a project for getting rain information; a problem the team were thinking about was how to get the system in as much of a ``set it and forget it" state as possible as it pertained to frost. The solution is to integrate with a weather service online and send notifications when there is a frost or freeze advisory.

After assembling the list of requirements, the team set out to create a high-level functional block diagram for each of the subsystems. This helps the team see where the different systems integrate for the future as well as breaking out all the different components that may need to be purchased.

The ultimate novelty in this project is all of the sensing that will be done through spectroscopy. The team has found a plethora of research on the topic and has started familiarizing themselves with the limitations and capabilities of the available technology. Ideally, the team would like to find a scalable solution to the sensing in which the optical sensing could be attached to a drone or satellite to survey fields for farming.
\section{Project Description}                   % Section 2
\subsection{Project Background}
Gardening is difficult. There are so many variables that a gardener must try to keep within their control. Soil moisture, temperature, soil composition just to name a few. The entire process of gardening is primarily uninvolved as a more involved process could be equated to the old adage ``watching grass grow.'' Here lies the issue, since plants grow without hardly any involvement from the gardener despite controlling the watering and feeding, a large part of the process can be automated.

There have been numerous ``Garduino'' projects that anyone could find and quickly modify to their needs. These projects generally use the sensors in the Arduino starter kits as well as an LCD or dot matrix display to read out the sensor data. Such a project is not well scoped for a senior design project. The difference in what we propose and what has already been done are the control aspects, automating watering and feeding and even temperature and sunlight regulation.

The sensors in the Arduino starter kits are all electrical sensors which in the high moisture environment of a plant bed will decay quickly. There has been a lot of progress in optical sensing and in our original preliminary research we found promising devices to sense soil moisture and composition optically instead of electrically. The hope is that such an optical sensor may one day be mounted on a drone and flown over a field to get the same data we are using to control our plant bed.

From the sensors, the project will also implement a control scheme. The hope is to build a gantry that supports a swiveling solar panel array as well as solenoids to control water. There are also items beyond the control of the plant bed such as weather (namely frost) which will be resolved in software by notifying the user of such conditions.
\subsubsection{Motivation}
The idea came from seeing the ``Garduino'' style projects all over hobbyist forums and websites but the idea really took hold in that each member of the team saw an opportunity to explore a new facet of engineering they held an interest in. This project provided the team an opportunity to apply our knowledge on power systems and delivery, controls, digital signal processing, and optical sensing. These are all areas that the team wanted to demonstrate a high level of understanding in and grow at the synthesis level.      % Section 2.1
\subsection{Project Objectives}
\subsubsection*{Goals and Objectives}
\subsubsection*{Function of Project}
\subsubsection{Project Specifications}
\subsubsection*{Requirements}
The MCU subsystem should:
\begin{itemize}
    \item Read local sensor data (e.g. sunlight, soil moisture, temperature)
    \item Adjust parameters of local modules (e.g. shade, water, nutrients)
    \item Interpret user settings and adjust parameters of modules accordingly
    \item Fulfill web requirements with at least two computers/controllers
\end{itemize}
It would be nice for the MCU subsystem to:
\begin{itemize}
    \item Fulfill web requirements with one computer/controller
    \item Have a local user display (e.g. LCD, dot matrix, segmented)
\end{itemize}
The direction of power in this system:
\begin{itemize}
  \item Power generated through a wall outlet
  \item AC/DC converter 
  \item Power supply
  \item MCU
  \begin{itemize}
    \item Regulators
    \item Sensors
    \item Mechanics
    \item CPU
  \end{itemize}
\end{itemize}
It would be nice to have:
\begin{itemize}
  \item Solar power, with the flow of power as follows:
  \begin{itemize}
    \item Solar panels 
    \item Solar power bank 
    \item Power supply 
    \item MCU
  \end{itemize}
  \item Use the solar panels as blinds to be able to open and close as well as collect energy
\end{itemize}
The Sensing Capabilities of the project should include:
\begin{itemize}
    \item Infrared Spectroscopy Sensor which detects:
    \begin{itemize}
            \item Soil Moisture
            \item Soil Temperature
            \item Soil OH group content (acidity)
    \end{itemize}
    \item Infrared Spectra signal processing
\end{itemize}
It would be nice to have:
\begin{itemize}
    \item Scanning Capabilities for checking various parts of the garden bed
    \item Soil Nutrient Estimation and other variables possible via IR Spectroscopy
\end{itemize}
The web component of the project should:
\begin{itemize}
    \item Attach to a weather API to receive:
    \begin{itemize}
        \item Rain
        \item Sun light
        \item Temperature
        \item Frost warnings
        \item Humidity
    \end{itemize}
    \item Alert users of conditions outside of automatic control (i.e. soil composition and frost)
    \item Change control parameters:
    \begin{itemize}
        \item Sun light
        \item Water
        \item Soil parameters
    \end{itemize}
    \item Have an intuitive user interface
    \item Communicate with the MCU
\end{itemize}
It would be nice to have the web component:
\begin{itemize}
    \item Set control parameters based on presets for plants
    \item Get plant data from the web to pass to MCU
    \item Communicate over secure channels
\end{itemize}
\subsubsection{Project Block Diagram}      % Section 2.2
\subsection{Marketing Requirements and Engineering Requirements}
Find our House of Quality figure below:
\begin{figure}[H]
    \centering
    \caption{House of Quality}
    \includegraphics[width=.45\textwidth]{images/HouseOfQuality.PNG}
    \includegraphics[width=.25\textwidth]{images/HouseOfQualityLegend.png}
\end{figure}    % Section 2.3
\section{Research}                              % Section 3
\subsection{Previous and Related Works}              % Section 3.1
\subsection{Related Technologies}
It is almost always beneficial for developers of a product to research and study related or similar products, as well as products related to their product's subsystems, in detail to gain a better understanding of how a subsystem can operate, and how subsystems can integrate with each other to form a cohesive system. In this section, members of our team explore technologies and solutions related to our product and its various subsystems. Some of the technologies studied below can be, and most likely will be integrated into our final product.

\subsubsection{Ocean insight: Ocean ST NIR Microspectrometer}
Ocean Insight is a local manufacturer of high-end, low size, weight, and power spectrometers. The ST NIR Microspectrometer is about 40 cubic centimeters in volume with a scan speed of 10ms, a signal to noise ratio of 190:1, and a spectral resolution of 2.2nm. Its spectral range is from 645nm to 1085nm, and it was specifically designed to be integrated into larger systems for customers who were interested in a flexible, low-cost design. Added to that, the system is rugged, and offers a variable slit input size, increasing its flexibility even further. While the designs are proprietary, this system serves as a benchmark for what can be achieved by the industry, and no doubt there are major design changes that can be made to achieve a similar result for the application intended for the Auto Garden Bed. That being said, the selling price for one is \textdollar1,750.

\begin{figure}[H]
    \caption{Ocean ST NIR Microspectrometer}
    \centering
    \includegraphics[width=0.5\textwidth]{images/3-2-1Pic.png}
\end{figure}

\subsubsection{AgroCares Nutrient Soil Scanner}

AgroCares offers a Near Infrared Spectrometer specifically designed for Proximity Soil Sensing. Its spectral range is from 1300 to 2500nm and it uses Micro Electrical Mechanical Systems or MEMS to capture EM Waves reflecting off the soil. The real value of the product is in its wireless communications system. The device uses Bluetooth 4.0 to send data to a cloud data center. There, spectrographs of large data sets of soil with known nutrient contents are compared with the reading, cutting out the need for on-sight calibration. The system is handheld and uses eight tungsten halogen bulbs to blast the soil with energy. This light is collected in an extremely small area, sampling 65 squared millimeters. It would be worth researching to see if the Tungsten bulbs were linked to the 1300 to 2500nm spectral range or if another probe and sensor would suffice.

\begin{figure}[H]
    \caption{AgroCares Nutrient Soil Scanner}
    \centering
    \includegraphics[width=0.5\textwidth]{images/3-2-2Pic.png}
\end{figure}

\subsubsection{DIY Webcam Diffraction Grating Spectrometer}\label{sec:DIYTransmissionGratingSpectrometer}

Physics Open Lab is a blog posting site for do-it-yourself physics laboratory projects. This project makes use of a megapixel webcam and a 1000 line/mm transmissive diffraction grating. The megapixel sensor allows for “staring” scanning, which is used in conjunction with spectrograph software for identifying the wavelength bands as they spread out from the zero order transmission. The project offers some interesting ideas, from the use of open source spectroscopy software to the use of two dimensional spectral analysis. Unfortunately, the use of a transmission grating presents a major problem for this application. Transmission gratings output most of their optical power straight ahead, and they are difficult to separate as higher order groups. While the project results where impressive, it did not generate a single spectrograph past the 1um range, which is necessary to detect the Moisture Content of the Soil.

\begin{figure}[H]
    \caption{Webcam Transmission Grating Spectrometer}
    \centering
    \includegraphics[width=0.5\textwidth]{images/DIYTransmissionGratingSpectrometer.png}
\end{figure}

\subsubsection{Proportional-integral-derivative Control}
Proportional-integral-derivative control (PID control) is a common control algorithm (espectially in industrial control systems) to allow a control loop to have reliable performance in a variety of conditions. Simply put, this algorithm allows a controller to receive an input and calculate a proportional output, accounting for error and rapid changes in the process. This algorithm may be useful to our application because it will allow the product to better control its facilities without user input.

The control function is defined in \autoref{eq:pid_controller}, where $u(t)$ is the control variable (e.g. the garden bed's water control solenoid), $K_p$, $K_i$, and $K_d$ are gain factors, and $e(t)$ (the error) is the different between a desired setpoint and measured process variable (e.g. the difference between the desired and current ounces of water dispensed).
\begin{equation}
    \label{eq:pid_controller}
    u(t) = K_pe(t) + K_i\int_{0}^{t}e(\tau) \, d\tau + K_d\frac{de(t)}{dt}
\end{equation}

\paragraph{Proportional Term} The proportional term of \autoref{eq:pid_controller} is $K_pe(t)$, hence referred to as the P-term. The P-term is proportional to the current error, and the gain $K_p$ determines the magnitude of the P-term. If the gain is too large, then the process variable will oscillate.

\paragraph{Integral Term} The integral term of \autoref{eq:pid_controller} is $K_i\int_{0}^{t}e(\tau) \, d\tau$, hence referred to as the I-term. This term accounts for previous values of $e(t)$ by taking the integral of the error, and the gain $K_i$ determines the magnitude of the I-term. The I-term aims to account for residual error in the control loop.

\paragraph{Derivative Term} The integral term of \autoref{eq:pid_controller} is $K_d\frac{de(t)}{dt}$, hence referred to as the D-term. The D-term aims to control future values of $e(t)$ by taking the derivative of the error, and the gain $K_d$ determines the magnitude of the D-term. This term acts to damp rapid changes in the control loop. Higher values of gain may make the control loop more sensitive to noise and lead to instability.

\subsubsection{Arduino Impletmentation of Microcontroller Internet Connection} Our team ultimately decided to implement a Texas Instruments microcontroller (detailed later) in the controls subsystem. This TI MCU contains an integrated network stack, and offers simple directions on connecting a compatible 2.4 GHz antenna. However, the one drawback our team did not expect was the relative abstractness, complication, and bureaucracy of creating programming with the tools required by TI. Between their proprietary distrobution of Eclipse, the libraries and SDKs required by the compiler, the configuratio of the compiler and linker, and the relative abstractness and complication of their libraries located in the SDK, the TI MCU is difficult to work with if the developer does not have pervious experience. In this section, Arduino's implementation is investigated and compared to our current TI MCU.

\paragraph{Arduino} Arduino is a microcontroller development board (integrating ATmega microcontrollers) distributor with a focus on hobbyists, especially entry-level developers. Compared to Texas Instruments, an industry-focused manufacturer, Arduino development has a very low bar to entry. Because of this, the fine granular control that may be required of an enterprise-level project is not present on Arduino boards---however, there are major advantages that make considering Arduino over TI worthwile:
\begin{itemize}
    \item Libraries are easy to implement
    \item The compiler and linker are relatively easy to configure compared to TI
    \item Large community following, support
    \item 3rd-party libraries are common and accessible
    \item Lower cost of components
    \item Easier to source components (in the year 2022)
    \item "Shields" (e.g. WiFi, GSM/GPRS, Bluetooth, GPS, motor controller, etc.) easily implementable
\end{itemize}
\paragraph{GSM/GPRS} One shield offered by Arduino is the \href{https://store.arduino.cc/products/arduino-mkr-gsm-1400}{Arduino MKR GSM 1400}, a 3G cellular network shield that enables SMS, voice, and internet connection. Arduino's library gives various "from scratch" examples and ample documentation on how to use different parts of its 1st-party \href{https://docs.arduino.cc/retired/archived-libraries/GSM}{GSM library}. For example, if one would like to connect a GSM network, it's as simple as including \texttt{GSM.h}, instantiating the \texttt{GSM} class (e.g. \texttt{GSM gsmAccess;}), and calling \texttt{begin()} (e.g. \texttt{gsmAcess.bein()}).

\paragraph{WiFi} Arduino offers shields like the \href{https://store.arduino.cc/products/arduino-mkr-wifi-1010}{Arduino MKR WiFi 1010}, as well as complete development boards like the \href{https://store.arduino.cc/products/arduino-uno-wifi-rev2}{Arduino Uno WiFi Rev2} with integrated WiFi modules. Just like the GSM library, Arduino's \href{https://www.arduino.cc/reference/en/libraries/wifi/}{WiFi library} is very easy to implement and use, and its documentation is wholly informative with class and function definitions and plenty of examples. For example, all one need do to connect to a network is detailed in the example in \autoref{fig:arduino_wifi_example}.

\begin{figure}[hbtp]
    \caption{Arduino WiFi code example}
    %\centering
    \label{fig:arduino_wifi_example}
    \begin{lstlisting}[language=c,frame=none,numbers=left,numbersep=-10pt,numberstyle=\tiny,basicstyle=\footnotesize\ttfamily]
        #include <WiFi.h>

        char ssid[] = "exampleNetwork";
        void setup() {
            while (status != WL_CONNECTED) {
                // Attempt to connect to SSID
                status = WiFi.begin(ssid);

                // Wait 10 seconds
                delay(10000);
            }

            // Once you reach here, you're connected
    \end{lstlisting}
\end{figure}

The user can go on to perform many different functions, with the limitation that they are either TCP or UDP bytestreams. This differs greatly from the TI CC3200's multifunctionality of being able to perform HTTP requests, Websockets, and many more on top of being able to take advantage of TCP and UDP bytestreams.                 % Section 3.2
\subsection{Part Selection}
\subsubsection{Controller Subsystem}
\begin{flushleft}
	At minimum, any chosen microcontrollers (MCUs) shall support natively, or by addition of a
	module, these features and traits:
	\begin{itemize}
		\item Analog-to-digital converter (ADC)
		\item cURL compatibility
		\item IEEE 802.11
		\item In stock and available to order
		\item JTAG module or equivalent
		\item Module communication bus (UART, I2C, SPI)
		\item Onboard CPU sufficient for our purposes
		\item Onboard memory sufficient for our purposes
		\item Onboard nonvolatile memory
		\item Pins dedicated to analog input
		\item Pins dedicated to digital I/O
	\end{itemize}
	These features would be "nice to have" on any MCU selected, but are not required:
	\begin{itemize}
		\item Digital-to-analog converter (DAC)
		\item microSD card slot
		\item Onboard battery
		\item Pins dedicated to pulse-width modulation (PWM)
		\item Timer(s) and an RTC
		\item USB compatibility
		\item Additional wireless communication protocols (e.g. BT or BLE, Zigbee)
	\end{itemize}
\end{flushleft}
\begin{flushleft}
	The selections, listed in \autoref{table:mcubreakdown1} and not in any particular order, match
	the above criteria and are being considered for selection.
	\begin{table}
		\centering
		\begin{tabularx}{\textwidth}
			{
				| >{\raggedright\arraybackslash}X
				| >{\raggedright\arraybackslash}X
				| >{\raggedright\arraybackslash}X
				| >{\raggedright\arraybackslash}X
				| >{\raggedright\arraybackslash}X
				| >{\raggedright\arraybackslash}X
				|
			}
			\caption{MCU option breakdown}
			\label{table:mcubreakdown1} \\
			\hline
			\textbf{Model} & \textbf{\href{https://www.ti.com/tool/LAUNCHXL-CC26X2R1}{LAUNCH\-XL-CC26X2\-R1}} & \textbf{\href{https://www.ti.com/tool/LAUNCHCC3220MODASF}{LAUNCH\-CC3220\-MODASF}} & \textbf{\href{https://www.raspberrypi.com/products/raspberry-pi-pico/}{Pico W}} & \textbf{\href{https://store-usa.arduino.cc/products/arduino-nano-33-ble?selectedStore=u}{Nano 33 BLE}} & \textbf{\href{https://www.st.com/en/evaluation-tools/b-l4s5i-iot01a.html}{B-L4S5I-IOT01A}} \\
			\hline
			\textbf{Manu\-facturer} & Texas Instruments & Texas Instruments & Raspberry Pi & Arduino & STMicro\-electronics \\
			\hline
			\textbf{Micro\-controller} & CC2652R & CC3220\-MODASF & RP2040 & nRF52840 & STM32\-L4S5VIT6 \\
			\hline
			\textbf{Processor} & 1x ARM Cortex-M4F & 1x ARM Cortex-M4 & 2x ARM Cortex-M0+ & 1x ARM Cortex-M4 & 1x ARM Cortex-M4 \\
			\hline
			\textbf{Maximum Speed (MHz)} & 48 & 80 & 133 & 64 & 120 \\
			\hline
			\textbf{Memory (KB)} & 256 ROM, 352 flash, 100 SRAM & 1024 flash, 256 RAM & 16 ROM, 264 SRAM & 1024 flash, 256 SRAM & 2048 flash, 640 RAM \\
			\hline
			\textbf{Wireless capability} & BLE5.2, Zigbee, Thread & 802.11b/g/n & 802.11n & BLE5.3, Zigbee, Thread, Matter & BT4.1, 802.11b/g/n, NFC \\
			\hline
			\textbf{Serial capability} & UART, I2C, I2S, SPI & UART, I2C, SPI & UART, I2C, SPI, USB1.1 & UART, I2C, I2S, SPI, USB2.0 & UART, I2C, SPI, USB2.0 \\
			\hline
			\textbf{Price (\$)} & 40, maybe free & 60, maybe free & 6 & 28 & 53 \\
			\hline
			\textbf{ADC} & 8-channel, 12-bit & 4-channel, 12-bit & 4-channel, 12-bit & 8-channel, 12-bit & 16-channel, 12-bit \\
			\hline
			\textbf{Clock capability} & Timer, RTC & Timer, RTC, WDT & Timer, RTC, WDT & Timer, RTC, WDT & Timer, RTC, WDT \\
			\hline
			\textbf{GPIO (pins)} & 31 & 29 & 30 & 13 & 16 \\
			\hline
			\textbf{PWM (channels)} & Supported & Supported & 16 & 4 & 6 \\
			\hline
			% \textbf{AES (bits)} & 128, 256 & 256 & Not supported & 128 & 128 \\
			% \hline
			\textbf{Required voltage (V)} & 1.8 -- 3.8 & 2.3 -- 3.6 & 1.8 -- 3.3 & 4.5 -- 21 & 4.75 -- 5.25 \\
			\hline
		\end{tabularx}
	\end{table}
\end{flushleft}
\begin{flushleft}
	Use of single-board computers (SBCs) was considered, but will not not need to be used; cURL 
	used on an MCU in conjunction with \href{https://aws.amazon.com/ec2/}{Amazon EC2} services will
	allow us to offload computing to a cloud solution.
\end{flushleft}
\begin{flushleft}
	Use of an external Wifi module is discouraged due to the following:
	\begin{itemize}
		\item Added cost
		\item Added complexity
		\item Modules in common use by hobbyists often have poor or no proper documentation, to the
		extent of:
		\begin{itemize}
			\item Quick start guide
			\item User's guide
			\item Datasheets
			\item Theory of operation
			\item Application uses
			\item Troubleshooting guide
			\item Schematics and mechanicals
			\item Quality and reliability
			\item Errata
		\end{itemize}
	\end{itemize}
\end{flushleft}
\begin{flushleft}
	Therefore, all of the MCUs listed above support either the 802.11 or Bluetooth standards.
\end{flushleft}
\begin{flushleft}
	Ultimately, the \href{https://www.ti.com/tool/LAUNCHCC3220MODASF}{LAUNCHCC3220MODASF}
	was chosen as the microcontroller development board for this project. In the event that the
	aforementioned LaunchPad is not able to be obtained, the
	\href{https://www.ti.com/tool/CC3220SF-LAUNCHXL}{CC3220SF-LAUNCHXL} has equivalent capability
	for the project's needs.
\end{flushleft}
\begin{flushleft}
	These boards, henceforth referred to as the CC3220, are able to be requested from our
	university at no upfront cost to our team. This was the driving factor behind choosing 
	the CC3220 over other microcontroller development boards. It was also determined that
	the microcontroller \emph{must} be able to interface via the 802.11 (WiFi) standard, for reasons
	that are detailed in \autoref{sec:controller_subsystem}---therefore, the LAUNCHXL-CC26X2R1
	and Nano 33 BLE were disqualified from selection. The Pico W was considered due to its low cost,
	and the B-L4S5I-IOT01A considered because of its abundant peripherals, but both ultimately lost
	out to the Texas Instruments products.
\end{flushleft}
% End Controller Subsystem

\subsubsection{Power Subsystem}
At minimum, this power subsystem will operate with the following:
\begin{itemize}
	\item Solar Panels 
	\item Rechargeable Batteries 
	\item Solar Charge Controllers
	\item AC/DC converterter
\end{itemize}
\textbf{Solar Panels}\par
The stretch goal for this project is to use solar panel arrays as blinds to increase/decrease sunlight as well as temperature. The solar panels are also use tocollect energy and power our model. We must be strategic when choosing our solar panels so that they are operational, provide the proper amount of power, and more.\par
There are many different types of solar panels. These include monocrystalline solar panels, polycrystalline solar panels, and thin-film solar panels. Each solarpanel has different compositions that make it as efficient as they are, how much power can be collected, etc. \par
For this project, we have selected multiple types of solar panels based on efficiency and cost.
\begin{table}[H]
    \centering
	
	\begin{tabularx}{\textwidth}
		{
			| >{\raggedright\arraybackslash}X
			| >{\raggedright\arraybackslash}X
			| >{\raggedright\arraybackslash}X
			| >{\raggedright\arraybackslash}X
			|
		}
		\caption{Solar panel types}
		\label{table:solarpanel} \\
		\hline
		\textbf{Solar Panel Type} & \textbf{Mono\-crystalline} & \textbf{Poly\-crystalline} & \textbf{Thin - Film} \\
		\hline
		\textbf{Efficiency} &  \textgreater20\% & 15 - 17\% & 6 - 15\% \\
		\hline
		\textbf{Power Rating} &  $\le$300W & 240 - 300W & Indefinite \\
		\hline
		\textbf{Performance} & Most efficient & Efficient & Least efficient \\
		\hline
		\textbf{Temperature} & High Tolerance & Low Tolerance & High Tolerance \\
		\hline
		\textbf{Cost per Watt} & \$1 - \$1.50 & \$.70 - \$1 & \$.43 - \$.70 \\
		\hline
	\end{tabularx}
\end{table}
From there we found these solar panels:
\begin{table}[H]
    \centering
	\caption{Solar panel part breakdown}
	\label{table:solarpanelparts}
	\begin{tabularx}{\textwidth}
		{
			| >{\raggedright\arraybackslash}X
			| >{\raggedright\arraybackslash}X
			| >{\raggedright\arraybackslash}X
			| >{\raggedright\arraybackslash}X
			| >{\raggedright\arraybackslash}X
			| >{\raggedright\arraybackslash}X
			|
		}
		\hline
		\textbf{Manu\-facturer Part \#} & P108 & P103C & P105 & SP-80X60-4-DK & SP-68X37-4-DK \\
		\hline
		\textbf{Manu\-facturer} & Voltaic Systems & Voltaic Systems & Voltaic Systems & AMX Solar & AMX Solar \\
		\hline
		\textbf{Dim\-ensions} & 10.9 x 8.8 x .16 & 8.27 x 4.46 x .2 & 5.39 x 8.74 x 0.16 & 3.15 x 2.362 x .079 & 2.677  x 1.456  x 0.079 \\
		\hline
		\textbf{Voltage at Pmpp} & 17.34V & 6.5V & 6.12V & 1.5V & 5.28V \\ 
		\hline
		\textbf{Current at Pmpp} & 570mA & 550mA & 940mA & 440mA & 69.3mA \\
		\hline
		\textbf{Open Circuit Voltage} & 20.45V & 7.7V & 7.13V & 1.8V & 6.27V \\
		\hline
		\textbf{Price (\$)} & 49 & 39 & 35 & 36.65 & 28.94 \\
		\hline
	\end{tabularx}
\end{table}
From there we needed to choose what kind of rechargeable battery we wanted to use for. Out on the market there are many different types of batteries including Nickel-Cadmium(NiCd), Nickel-Metal Hydride(NiMH), Lithium Ion(Li-Ion), and so many more.\par
Of these batteries, we decided to go with the Lithium Ion battery because it was the most commonly used battery for electronic devices while allowing high output voltage.
\begin{table}[H]
    \centering
	\begin{tabularx}{\textwidth}
			{
			| >{\raggedright\arraybackslash}X
			| >{\raggedright\arraybackslash}X
			| >{\raggedright\arraybackslash}X
			| >{\raggedright\arraybackslash}X
			| >{\raggedright\arraybackslash}X
			|
		}
		\caption{Battery investigation}
		\label{table:battery} \\
		\hline
		\textbf{Manu\-facturer} & \textbf{Ampere Time} & \textbf{Expert\-Power} & \textbf{Eco Worthy} & \textbf{Eco Worthy} \\
		\hline
		\textbf{Voltage} & 12 & 12 & 12 & 12 \\
		\hline
		\textbf{mAh} & 6000 & 10000 & 5000 & 8000 \\
		\hline
		\textbf{Price (\$)} & 29.99 & 59.99 & 35.99 & 43.99 \\ 
		\hline
	\end{tabularx}
\end{table}
For the solar panels, having a solar charge controller is important to the system. The purpose of the solar charge controller is to optimize the charging of the battierys by the solar panels. There are two major types of solar charge controllers: Maximum Power Point Tracking (MPPT) and Pulse Width Modulated (PWM). \par 
With these two in mind we chose these charge conrollers:
\begin{table}[H]
    \centering
	\begin{tabularx}{\textwidth}
			{
			| >{\raggedright\arraybackslash}X
			| >{\raggedright\arraybackslash}X
			| >{\raggedright\arraybackslash}X
			| >{\raggedright\arraybackslash}X
			| >{\raggedright\arraybackslash}X
			|
		}
		\caption{Charge Controller}
		\label{table:chargecontroller} \\
		\hline
		\textbf{Manu\-facturer} & \textbf{Expert\-Power} & \textbf{Expert\-Power} & \textbf{Renogy} &  \textbf{Renogy} \\
		\hline
		\textbf{Nominal Voltage} & 12\slash24V  & 12\slash24V & 12\slash24V & 12\slash24V \\
		\hline
		\textbf{Rated Charge Current} & 10A & 20A & 10A & 30A \\
		\hline
		\textbf{Max PV Input Voltage} & 50V & 100V & 50V & 50V \\
		\hline
		\textbf{Self Consumption} & $\leq$10mA & $\leq$10mA & \textless10mA & $\leq$13mA \\
		\hline
		\textbf{Price (\$)} & 23.99 & 69.99 & 34.99 & 69.99 \\ 
		\hline
	\end{tabularx}
\end{table}
While the model is charging during the day or as a back up, this will be plugged into a wall outlet for power. This means we have to be able to convert the AC voltage coming from the wall is converted to DC voltage for the model to use.
\begin{table}[H]
    \centering
	\begin{tabularx}{.8\textwidth}
		{
			| >{\raggedright\arraybackslash}X
			| >{\raggedright\arraybackslash}X
			| >{\raggedright\arraybackslash}X
			| >{\raggedright\arraybackslash}X
			|
		}
		\caption{AC/DC Converter breakdown}
		\label{table:acdcconverter} \\
		\hline
		\textbf{Manu\-facturer} & \textbf{SmoTecQ} & \textbf{ANLINK} & \textbf{TMEZON} \\
		\hline
		\textbf{Input Voltage} &  240V & 100 - 240V & 100 - 240V \\
		\hline
		\textbf{Output Voltage} & 12V & 12V & 12V \\
		\hline
		\textbf{Current Rating} & 2A & 2A & 2A \\
		\hline
		\textbf{Connector} &  5.5 mm x 2.1 mm &  5.5 mm x 2.1 mm &  5.5 mm x 2.1 mm\\
		\hline
		\textbf{Price (\$)} & 12.99 for 2 & 11.59 & 8.99 \\ 
		\hline
	\end{tabularx}
\end{table}

\subsubsection{Sensing Subsystem}
The required sensing subsystem capabilities include:
    \begin{itemize}
        \item Diffraction Grating or Diffraction Wheel
        \item Motorized frequency selection mechanism
        \item SWIR regime Photodiode
    \end{itemize}

    Frequency separation mechanisms under consideration are:
    
\begin{table}[H]
\begin{tabularx}{\textwidth}
    {
        | >{\raggedright\arraybackslash}X
        | >{\raggedright\arraybackslash}X
        | >{\raggedright\arraybackslash}X
        | >{\raggedright\arraybackslash}X
        | >{\raggedright\arraybackslash}X
        | >{\raggedright\arraybackslash}X
        |
    }
	\caption{Infrared Sensors} 
	\label{table:sensors}\\
    \hline
    \textbf{Model} & \textbf{\href{https://www.edmundoptics.com/p/ingaas-detector-70mum-dia-to-46/12571/}{0.07mm Dia., TO-46 Package, InGaAs Photodiode}} & \textbf{\href{https://www.edmundoptics.com/p/ingaas-detector-120mum-dia-to-46/12574/}{0.12mm Dia., TO-46 Package, InGaAs Photodiode}} & \textbf{\href{https://www.thorlabs.com/thorproduct.cfm?partnumber=FGA01}{FGA01 - InGaAs Photodiode}} & \textbf{\href{https://www.thorlabs.com/thorproduct.cfm?partnumber=FGA015}{FGA015 - InGaAs Photodiode}} \\
    \hline
    \textbf{Manu\-facturer} & Edmund Optics & Edmund Optics & ThorLabs & Thorlabs \\
    \hline
    \textbf{Spectral Response (nm)} & 900 - 1700 & 900 - 1700 & 800 - 1700 & 800 - 1700 \\
    \hline
    \textbf{Active Area Diameter (mm):} & 0.07 & 0.12 & 0.01 & 0.018 \\
    \hline
    \textbf{Responsivity (A/W)} & 0.9 & 0.9 & 1.003 & 0.95 \\
    \hline
    \textbf{Price (\$)} & 88 & 88 & 67.55 & 63 \\
    \hline
	\end{tabularx}
\end{table}

Diffraction gratings and LED Probes are still being evaluated for necessity:             % Section 3.3

\section{Design Constraints}                    % Section 4
\subsection{Related Standards}
\subsubsection{C++14} C++ will be programmed according to the C++14 standard provided by Texas Instruments' ARM compiler. This standard is formally known as
\href{https://www.iso.org/standard/64029.html}{ISO/IEC 14882:2014}. C++ is
a superset of C, and builds upon it by introducing object-oriented
programming concepts while maintaining the functional language aspect of C.

\subsubsection{802.11} The microcontroller (MCU) supports transmission through the Institute of
Electrical and Electronics Engineers (IEEE) 802.11b/g/n standard of wireless
communication. This standard uses the S band of radio frequences and
operates at 2.4 GHz. There are 14 accessible channels, each spanning a band
width of 22 MHz (pictured in \autoref{wifi_channels}).
\begin{figure}[H]
    \caption{802.11b/g/n channels}
    \label{wifi_channels}
    \centering
    \includegraphics[width=\textwidth]{images/wifi_channels.png}
\end{figure}
These channels specifically reside in an industrial, scientific and medical
(ISM) band. This standard also provides datagram frames for the transport
layer.

\subsubsection{TCP} \label{tcp_standard} Transmission Control Protocol (TCP) will be used to satisfy transport layer
requirements of the product, and will be used to transmit symbols (i.e.
from any commands, data, settings, telemetry, etc.) between Amazon Web
Services (AWS) and the microcontroller (MCU). TCP was chosen over other
protocols, such as User Datagram Protocol (UDP), mainly due to its
reliability. The extent of TCP's reliability includes features such as
checksums, duplicate data detection, retrying of transmissions, sequencing,
and timers.
Such reliability is favored over higher bandwidth or lower
latency, as neither of the latter are required for the kilobytes
of information being relayed between AWS and the MCU. A standard TCP frame
is shown in \autoref{tcp_frame}.
\begin{figure}[H]
    \caption{TCP frame (\href{https://condor.depaul.edu/jkristof/technotes/tcp.html}{The Transmission Control Protocol}, Fig. 1)}
    \label{tcp_frame}
    \centering
    \includegraphics[width=\textwidth]{images/tcp_frame.jpg}
\end{figure}


\subsubsection{IPv4} IPv4 is the fourth version of the Internet Protocol, a network layer protocol in use to relay data between devices and across networks. The data relayed, datagrams, are sent between sources and hosts that are identified by 32-bit addresses. This protocol strictly functions to transport the datagram from one device to another, with no end-to-end reliability, flow control, sequencing, or other measures found in other protocols such as TCP. IPv4 provides two distinct features: fragmentation of whole datagrams, and addressing of devices. A standard IPv4 frame is shown in \autoref{ipv4_frame}.
\begin{figure}[H]
    \caption{IPv4 frame (\href{https://dx.doi.org/10.17487/RFC0791}{RFC 791, Internet Protocol, DARPA Internet Program Protocol Specification}, Fig. 1)}
    \label{ipv4_frame}
    \centering
    \includegraphics[width=\textwidth]{images/ipv4_frame.png}
\end{figure}

\subsubsection{JSON Web Token (RFC 7519)}
This standard specifies a ``compact, URL-safe means of representing claims to be transferred between two parties.'' The means is via the JSON Web Token (JWT). JWTs are split into 3 parts, the header which specifies the algorithm the key(s) used to encrypt the message and the type of token, JSON Web Encryption (JWE) or JSON Web Signature (JWS). The second part is the payload. The payload is a JSON formatted object which carries claims. And the third part is the verification signature which is the Base64 encoded header plus a `.' plus the Base64 encoded payload, another `.' and finally this is all encrypted by the key. When a JWT is received, the header and payload can be read by just Base64 decoding these portions of the token. The JWT is verified by decrypting the verification signature, if the signature cannot be decrypted with the key then the token is invalid. \autoref{fig:jwt} gives an example of a JWT that is signed with a secret key.
\begin{figure}[H]
\centering
\caption{Example JWT from jwt.io}
\includegraphics[width=\textwidth]{images/jwt.png}
\label{fig:jwt}
\end{figure}

\paragraph{Claims}
Claims are in the payload of a JWT. The basic claims specified by this standard are \verb|iss| ``issuer'', \verb|sub| ``subject'', \verb|aud| ``audience'', \verb|exp| ``expiration'', \verb|nbf| ``not before'', \verb|iat| ``issued at'', and \verb|jti| ``JSON Token ID''. For our purposes, all of the JWTs will be self-signed so the issuer, JSON Token ID, audience, and issuer claims can all be disregarded as they are meant for cross-service authentication. However, following the standard, the ``subject'' claim will be the user principal, and the expiration and issuad at claims will be used to invalidate a token cookie after a certain period of time. The standard allows for public and private claims outside of these as well. Public claims are registered with the IANA while private claims are collision prone, neither of these are of true concern to us.
\subsubsection{HTTP/1.1 (RFC 2616)}
This standard defines an application-level protocol communicating across the internet. RFC 2616 is an update to previous protocols that increase the capabilities of the original in the form of persistent connections and the ability to send files in the form of MIME types. MIME types hold metadata and the binary data of a file. This standard also serves to define how dates/times are handled on the web as well.

\subsubsection{WebSocket Protocol (RFC 6455)} \label{websocket_protocol}
The WebSocket Protocol standard (aka. "WebSockets") is an application layer protocol that allows full-duplex communication over a single TCP connection. WebSockets can be considered a comparable to HTTP, but is compatible with HTTP, though it functions effectively similarly to TCP serial bytestreams, with the advantage in using WebSockets being in that devices connect via a URL. This allows devices to rely on on URLs for addressing instead of statically-assigned IP addresses. The WebSocket Protocol also further specificies how the connection is made, how the data is formatted, and some security considerations. Most of these items are abstracted away from users in the forms of libraries---however, the security concerns are likely the most important part of this standard in terms of this project. This reason being that we are receiving location and IP data from the packets in order to build out commands like where to point the solar panels. Exposing PII such as location to the world wide web is a major concern that we would like to avoid by following this standard.        % Section 4.1
\subsection{Constraints}
The following section nwill cover specific limitations, or constraints, the team has and will face during the project timeline. These restraints have affects on our ability to design specific syst4ems and limits our options when designing different parts of the product. The following are our constraints: economic, time, equipment, safety, environmental, manufacturability, ethical, and sustainability. Each section will discuss possible and realized problems with regard to the specific category of constraint. After explaining and detailing each constraint, we will expolore different solutions that we could use to overcome any barriers. Some problems may not occur for the team, but we want to be prepared for any constraints that may arise. 
\subsubsection{Economic}
The first major constraint that we expected and have already begun to face is economic. We considered and looked for different sponsorhip options for this project, but unfortunately our team was unable to secure a sponsorship. Our funding for the project will be entirely self-funded and evenly divided between the four team members. As a result of being funded entirely out-of-pocket, cost has been and will continue to be a driving factor in the materials and components our team decides to use for the product. When analyzing various parts and materials for the product, ideally, we would choose the best available, but when budgeting for both us and the consumers we plan to serve, we plan to maximize specifications and requirements while minimizing the cost. This means our product may not have all the "nice-to-have" features and will instead have features to make it perform just enough to be competitive in the market we are designing the product for.  \\

Another hurdle that the economy brings to this project is the availability of certain parts. Manufacturers across the world have been drastically impacted by the economy over the last few years making nvarious products unavailable, hard to get, or more expensive than ever before. We excpect to spend more time comparing components because of lack of availability compared to the pricing of such parts. Lack of availability of certain components will also affect the final outcome of our product because we may not be able to get the exact parts we want as part of our design to meet time requirements for the project.\\

We have run in to the problem of availability with products on multiple occasions. An example of this is when we were looking for charge controllers from the manufacturere Analogue Devices. We were originally looking at the charge controller "Power Tracking 2A Battery Charge for Solar Power (LT3652), but quickly found that it was unavailable. This specific option for a charge controller was an ideal option because of its pricepoint and specifications. Our product only needs 5V to operate so when looking for a component we were looking to meet just about our needs. Our second choice was through TI which had a minimum of 5V and max of 28V, but again ran in to the issue that the component was not available. We are now considering two other options that are more expensive, have higher voltage, and offer other features such as tilemetry, and a low-loss power path that we do not need for our product.
%to-do: lo-loss power path in last sentence is a trademark. section below this comment is a part of the paragraph above. new paragraph begins with "The side effects"

Not only are we being forced to select something more than needed, but this will cause future restraints on components remaining in budget and has delayed time because of the extra time spent searching for products available that meet our requirements.\\

The side effects of our budget and product availability in the economy is also affected by the inflation rates. Not only could we run in to the problem of not having a product readily available to meet our needs, but inflation could also cause a component to fall outside of our budget. \\

In order to combat this restraint, we have looked at future products we may need and have been considering differnt options ahead of time. This will allow us to plan ahead and order products for future parts of the project now so we do not have to wait on them later. \\
\subsubsection{Time}
Most projects share several different constraints. One of the constraints that will be found with every product is the restraint of time. Customers always want the next product and teams need various resources made on a deadline. For this project, we are on a timeline to complete the project, and various iterations, with the semester. Most projects revolve around the deadline that meets a customers' needs. In this case, the project is part of a larger organization that runs through business cycles every year. The deadline in place is imposed by the restructuring of the university as a workforce.\\

Not only is the project restricted on a semesterlyn timeline, but the project also relies on the schedule set forth by the university. We recently experienced a time constraint when a hurricane came through Orlando. Not only did the university close, but some of our team members lost power and were unable to work during this time. Additionally, the university will close over the holidays limiting resources and ability to make some decisions to progress the project. \\

As all parts of this project correlate with each other, so do our restraints. Ealier we discussed that some products have limited, or no, availability, This further pressurizes the time restraint we have because of product delays. We have already been delayed by certain products being available and have polans to look further into our project design to try and avoid further delays with shipping and manufacturing. \\

Another factor that affects the timing of our project, is our own availability. Each of our team members have various projects and responsibilities, and we work together to find time that works for everyone on the team. Sometimes our schedules do not align and causes us to push our meetings back a day or two. This is something we have been working around by communicating regularly through Discord even if we have not been able to meet face to face.\\

Since we know we have another semester working on this project, we have had more flexibility this semester to move things where needed in our project timeline. The closer we get to the deadline, the less flexibility we will have with our time. Because we recognize this restraint now, we have discussed this topic and allowed for time to make changes in the future by making as much progress now as possible.\\ 

\subsubsection{Equipment}
There are three main equipment constraints on our ability to produce our project. One is limited access to software, another is heavily shared tools, and the third is having facilities to build.

This project arises out of a tradition of problem solving that is used in educational facilities and industries all over the country. A chief part of the College of Engineering and Computer Science program is to teach students to use the tools industry uses to solve the problems industry faces. So far, our team has used multiple integrated development environments, version control platforms, word processers and file viewers, text, video and audio communication services, and software for designing optical beam paths, CAD models of physical structures, and PCB simulations in order to express and determine design features for our project. Each of us has had tools that we would like to have used, but could not, either because of licensing, inexperience, or the need to work cooperatively with the larger team. 

Another equipment constraint has been in the form of shared workspaces. In order to leverage these tools, we need relatively quiet space to sit where we can communicate with each other and run power hungry devices while connected to the internet and organizational networks. Our University has provided us with design labs, but their tools are managed by students, and inventory is often untraceable and disorganized. 

The third equipment constraint is unique to the nature of this project, but it comes from needing to build outdoors. Building a garden bed requires land, if only a little, and although parts can be fabricated in clean spaces the system is designed to contain wet, heterogeneous dirt. Space outdoors is necessary to build and that means either permission from the University or leveraging team member access to land. 

Each of these equipment constraints has a different effect on the project, some of them are easy to get around, by finding open source alternatives to licensed software, by cooperating with other students to get more out of leftover optical components, and by using student networks to determine where and when our group will have space to meet. Some are more difficult, causing us to trade time and money that we would rather not have had to give. In the end, equipment constraints will not likely impact the project deadline or affect the features as laid out in the document.

\subsubsection{Safety}
This project features several threats to human safety that must be addressed. The power subsystem is rated with sufficiently high voltage and current to cause cardiac arrest. The user must be protected from direct exposure to electrical conductors within the system, especially as this is an outdoor system with water management mechanisms which may add to the risk. The optical subsystem involves the use of infrared probes and focusing lenses. If handled carelessly, these could potentially create an eye hazard during testing or product use that the victim would not be able to detect. Steps must be taken to indicate the nature of the threat where it exists. Other risks involved in the project include the chance of a minor cut on a sharp surface becoming infected from exposure to the soil. The structure and components of the project must be sufficiently dull to ensure against the possibility of this. \\

Recognizing the improtance of our team's safety now allows us to think through different precautions we can take when building and testing the product. One precaution our team plans to utilize is personal protection equipment and understanding the tools we will utilize. We will also ensure that we are using proper tools for different jobs. We will discuss as a team with each other to know what tools we have and potential items needed to purchase to ensure we have the right equipment for the job. Taking inventory of this now will also allow us to budget any tools in as needed. We will also plan accordingly on when to bring professionals in for various needs. For example, no one on our team is trained to cut and bend the garden materials on the extorior so we plan to utilize campus resources and personal connections to ensure we get the job done correctly and safely. \\

\subsubsection{Environmental}
Our product features our consumers utilizing a garden bed which directly engages with the environment. Because our product is doing this, there are environmental factors that our team must consider. Although we are not directly pouring soil into each customer's garden bed, we know that different fertilizers will be purchased and used to fill our products. This has potential of disturbing ecosystems, even if it was through the butterfly effect. Additionally, we will need to consider the risk of pollution our batteries and materials cause on the environment. When we discuss the marketing requirements, we also consider the environmental impact because we know our consumers will care about this factor. We also have to consider the fertilizer runoof, sound pollution, and light pollution that could occur as a result of our project.\\

To consider the impact we could have on the economy, we are also taking careful consideration in the components and materials we will use for our product. The component that will have one of the larger impacts on this rest5raint is our battery. Many of the other components and parts can be re-used, but batteries do not have the same type of use after the product is retired. This is also a reason we decided to make our product solar. We want it to last as long as possible and use less energy where it can. Not only are we considering performance standards of various components, but also the lasting impact these aprts will have on our environment for years following. The logner we can make our product last and the better, the less that will negatively impact the environment. 


\subsubsection{Manufacturability}
Manufacturability is a set of important criteria for early in the design process to avoid making costly mistakes. When originally selecting our project, we wanted to make sure that whatever project we selected would be achievable and able to be built. For this project, we had to ensure that building the garden bed would be something feasible for our team and the resources we had available to us. We considered the two reasons that could cause a product to potentially not be manufacturable. The first one is that the design is too complex to be completed. This could be caused by any number of reasons within the design. We considered all the other constraints that have been discussed and whether those constraints would further constrain nthe manufacturability of the garden bed. The second reason relates back to the cost restraint. We had to budget and determine if we would be able to fund all the requirements to build out the entire product, and account for variability. \\

Our team has protected against cost prohibitive manufacturability issues by ensuring that our component selection is traceable and that each component is in production and within our price range at cost, not due to clearance discounts. Another step our team is taking against future constraints is selecting the bulk of our components and design features before taking steps toward production.  

\subsubsection{Ethical}
When considering the ethics behind any project, this project made our ethical constraint easier. We know that the negative impact will be outweighed by the positives our design brings to any market. Not only does our product encourage consumers to grow their own vegetables, or plants, it will also bring a mindfulness to many lives. There are countless studies that show the benefits of being outside and tending to a garden, and this project enables that behavior. Our garden bed will encourage age old activities that do not threaten the well being of the user or the surrounding community. Our project is designed to accommodate and minimize its environmental impact. Beyond the environmental ethics, we also considered user data and security of the product. Because of the wireless communication system we are designing and the information it needs to work, we have drastically limited security risks for the consumer and surrounding neighbors. There is also minimal room for any harm to come to the consumers. Consumers that engage with the product will not be able to use it to harm others, knowindly or unknowingly. The product is also open to any consumer to utilize and does not risk damaging ay cultural heritage or societies. The garden bed can be used by anyone who decides to purchase it.  Our team commits to submit to University policy as regards the goals and activities of this project.

\subsubsection{Sustainability}
Sustainability means creating a system that will produce as much of the resources as it uses so that the system does not collapse. An example of this is when we were deciding what materials to make the structure out of. We wanted to maximize life of the product without hindering our budget. We found that aluminum meets several consumer requirements, as well as our other project restraints. Aluminum is also more sustainable than some other materials, such as wood, that were involved in the discussion. Another point for sustainability that we have considered is both battery life and the life of the solar panels. We want to find a balance between the life and cost of these components. Additionally, we hope to have them integrated in a way that when they need to be replaced, the consumer is not forced to replace the entire product, but only the ones that are beyond their useful life. \\

Beyond product life sustainability, we also considered other factors. The power generation subsystem is intended to ensure that our system does not need to draw on the power grid in order to function correctly. This will lower the impact on both the environment and how well the product can sustain itself. We also considered the water detection to ensure accuracy so that water is not wasted. The sensors and system will minimize water waste by only watering when needed and at appropriate times before or after sunshine. \\

In a broader sense, our project will increase the wellbeing of people’s lives, producing new plants and possibly food. One obstacle to sustainability is power storage. Our system uses a solar panel to collect energy, but that energy needs to be stored in order to distribute it in the right amounts and at the right times. We will be selecting our battery carefully and minimizing the waste of this as much as possible.
 % Section 4.2-4.9
\section{System Hardware and Software Design}   % Section 5
High level overview of design
\subsection{Controller Subsystem}
\label{sec:controller_subsystem}
\begin{figure}[H]
    \label{mcu_block_diagram}
    \caption{MCU block diagram}
    \centering
    % Need to update block diagram
    \includegraphics[width=0.75\textwidth]{images/mcu_block_diagram.png}
\end{figure}
% I need to write 11 pages lmfao
% 4 more pages (page total 27)
\begin{flushleft}
    % Why the MCU connects to the internet
    In order for our system to be as self-sufficient and power efficient as
    possible from an end-user perspective, it was determined that our system
    would require an internet connection to offload remote command-and-control
    to an Amazon Web Services EC2 instance (hence referred to as "AWS" and detailed in
    \autoref{sec:web_subsystem}). To make the process of operating our product 
    as hands-off as possible to end-users, the microcontroller will connect to
    the user's home WiFi network for access to AWS. Bluetooth, Zigbee, Thread,
    and other short-range 2.4 GHz communication protocols were disfavored over
    WiFi, as we predict most users will not have a device to dedicate to
    connecting our product via such protocols. Long range (LoRa) protocols were
    deemed unncessary, as the intended placement of our product is outside, 
    near or next to the user's home. We do not expect our product to produce
    or receive large amounts of data, so the decreased bandwidth of a
    WiFi-enabled product being beyond the outdoor walls of a building
    is not a significant drawback to our application. A wired connection
    (802.3/Ethernet) was deemed too invasive to the end-user. It is expected
    that most, if not all, end-users have a wireless access point and internet
    access. Thus, connection via the 802.11/WiFi standard was a natural choice
    for our use case.
    % maybe diagram of chip or network stack?
\end{flushleft}
\begin{flushleft}
    % How the MCU connects to the internet (local network, LAN -> NAT -> WAN,
    % TCP stack)
    The Texas Instruments CC3220-series of microcontrollers are WiFi-enabled
    chips with an ARM Cortex-M4 central processor and a WiFi network processor,
    along with many useful peripherals and power management modules. The WiFi
    network processor supports the following standards/features useful to our
    development:
    \begin{itemize}
        \item WiFi standards: 802.11b/g/n
        \item WiFi security: WEP, WPA/WPA2 PSK, WPA2 enterprise, WPA3 personal,
        WPA3 enterprise
        \item WiFi provisioning: SmartConfig, WPS2
        \item IP protocols: IPv4, IPv6
        \item IP addressing: static IP, DHCPv4, DHCPv6
        \item Transport: UDP, TCP, RAW
        \item Host interface: UART, SPI
        \item Built-in transceiver and 2.4 GHz antenna
    \end{itemize}
    Our microcontroller will 
    % WHAT KIND OF INTERFACE????
    % option a) web interface for user to long in to WLAN
    % option b) WPS button
    % don't forget IP addressing (most likely DHCP)
    % Interim: hardcode network login
\end{flushleft}
\begin{flushleft}
    % TCP sockets
    The MCU will communicate with AWS through TCP sockets. After connection to
    the user's home network, the MCU will check if there is an internet
    connection. Once an internet connection has been established, the MCU will
    open a socket and connect to the AWS instance via its URL (using the
    default DNS nameserver provided to network clients).
\end{flushleft}
\begin{flushleft}
    % Parameters for connection and how often it tries
    Once the MCU has established a connection to the internet and the AWS
    instance, it attempts to send current system settings and telemetry, as
    well as sensor readings to AWS. The microcontroller does this at least
    once every 15 minutes. Sending and receiving may occur more often if
    commanded to by AWS. Because TCP is being used to connect AWS and the
    microcontroller, any manual retries on a failed send or receive most
    likely will be futile. Therefore, any sort of link error handling will
    be performed by the link and not the microcontroller program.
\end{flushleft}
\begin{flushleft}
    % Any over-the-air updates?
    At this time, our team does not intend to provide a method for over-the-air
    updates (OTA), however, this is a provision that may be developed in the
    future.
\end{flushleft}
\begin{flushleft}
    % MCU receives commands, decodes commands
    The MCU will receive commands and data from AWS in the following form:
    \begin{center}
        \texttt{t x c y}
    \end{center}
    where \texttt{t} (character) marks the beginning of the command string,
    \texttt{x} (unsigned integer) indicates how many seconds to wait before
    executing command \texttt{c} (character) with parameter \texttt{y}
    (ambiguous). The following characters occupy the spot of \texttt{c}, and
    translate to the following commands:
    \begin{center}
        \texttt{w y}: water, \texttt{y} is boolean \\
        \texttt{a y}: solar $\theta$, \texttt{y} is double \\
        \texttt{b y}: solar $\phi$, \texttt{y} is double \\
        \texttt{l y}: low power, \texttt{y} is unsigned int for flags \\
        \texttt{s y}: send data, \texttt{y} is unsigned int for flags
    \end{center}
    For example, if AWS instructs the MCU to turn on the water source in 3
    minutes, it would send the command \texttt{t 180 w 1}. If AWS would like
    the MCU to change the $\theta$ of the solar panels to 45$\degree$
    immediately, it would send \texttt{t 0 a 45.0}. It is important to note
    that the values of \texttt{x} and \texttt{y} are not strings (e.g.
    45$\degree$ will not be sent as \texttt{"45.0"}), but the actual encoding
    of 45$\degree$ as a double-precision float. This is to maintain a
    consistent command string size amongst all transmissions.
    \begin{figure}[H]
        \label{command_bitwise}
        \caption{Bitwise representation of command}
        \centering
        \includegraphics[width=0.75\textwidth]{images/command_encoding.png}
    \end{figure}
\end{flushleft}
\begin{flushleft}
    % Will program in C++. Any libraries, multithreading?
    % Singleton Pattern Thread for scheduling tasks
    It should be expected that all headers from the C++ standard library (as
    defined in C++23) will be used, along with the following nonstandard
    libraries: Texas Instruments SimpleLink™ CC32xx SDK. Additionally, it is
    expected that the MCU program will schedule tasks using a Singleton Pattern
    thread to ensure thread safety when accessing variables.
\end{flushleft}
\begin{flushleft}
    % Classes and class diagram
    There will be a few classes defined in the MCU's programming.
    \begin{figure}[H]
        \label{classes_uml}
        \caption{UML diagram of the classes used}
        \centering
        \includegraphics[width=0.75\textwidth]{images/classes_uml.png}
    \end{figure}
    Classes will be laid out in a header file using the following defintions:
    \begin{flushleft}
        \texttt{class Plant \{}  \\
        \quad\texttt{private:} \\
        \quad\quad\texttt{bool water\_active;} \\
        \quad\quad\texttt{double temp;} \\
        \quad\quad\texttt{double moisture;} \\
        \quad\quad\texttt{uint32\_t oh\_comp;} \\
        \quad\quad\texttt{uint32\_t nutrients;} \\
        \quad\texttt{public:} \\
        \quad\quad\texttt{Plant();} \\
        \quad\quad\texttt{$\sim$Plant();} \\
        \quad\quad\texttt{set\_water\_active();} \\
        \quad\quad\texttt{get\_water\_active();} \\
        \quad\quad\texttt{set\_temp();} \\
        \quad\quad\texttt{get\_temp();} \\
        \quad\quad\texttt{set\_moisture();} \\
        \quad\quad\texttt{get\_moisture();} \\
        \quad\quad\texttt{set\_oh\_comp();} \\
        \quad\quad\texttt{get\_oh\_comp();} \\
        \quad\quad\texttt{get\_oh\_comp\_raw();} \\
        \quad\quad\texttt{set\_nutrients();} \\
        \quad\quad\texttt{get\_nutrients();} \\
        \quad\quad\texttt{get\_nutrients\_raw();} \\
        \texttt{\};} \\
    \end{flushleft}
    \begin{flushleft}
        \texttt{class Solar \{}  \\
        \quad\texttt{private:} \\
        \quad\quad\texttt{double theta;} \\
        \quad\quad\texttt{double phi;} \\
        \quad\quad\texttt{double voltage;} \\
        \quad\texttt{public:} \\
        \quad\quad\texttt{Solar();} \\
        \quad\quad\texttt{$\sim$Solar();} \\
        \quad\quad\texttt{set\_theta();} \\
        \quad\quad\texttt{get\_theta();} \\
        \quad\quad\texttt{set\_phi();} \\
        \quad\quad\texttt{get\_phi();} \\
        \quad\quad\texttt{set\_voltage();} \\
        \quad\quad\texttt{get\_voltage();} \\
        \texttt{\};} \\
    \end{flushleft}
    \begin{flushleft}
        \texttt{class Web \{}  \\
        \quad\texttt{private:} \\
        \quad\quad\texttt{string ip;} \\
        \quad\quad\texttt{int port;} \\
        \quad\texttt{public:} \\
        \quad\quad\texttt{Web();} \\
        \quad\quad\texttt{$\sim$Web();} \\
        \quad\quad\texttt{set\_ip();} \\
        \quad\quad\texttt{get\_ip();} \\
        \quad\quad\texttt{set\_port();} \\
        \quad\quad\texttt{get\_port();} \\
        \quad\quad\texttt{receive();} \\
        \quad\quad\texttt{send();} \\
        \texttt{\};} \\
    \end{flushleft}

    % Turn structs into classes
    % Get rid of telemetry
    % Implement default state
\end{flushleft}
\begin{flushleft}
    % Send to AWS, format and types of commands
    % Send straight data from the struct
    Unlike commands received from AWS, the MCU will simply send data directly
    from its classes to AWS. This is done to minimize the overhead of sending
    extraneous symbols and preserve power stored in the battery, as receiving
    uses far less power than transmitting in the MCU subsystem. Further
    optimization can be performed in the future to further reduce overhead
    (e.g. reducing \texttt{water\_active} to occupy 1 bit and using the rest of
    the symbol to encode other data).
    \begin{figure}[H]
        \label{toweb_encoding}
        \caption{Bitwise representation of data sent to AWS}
        \centering
        \includegraphics[width=0.75\textwidth]{images/toweb_encoding.png}
    \end{figure}
\end{flushleft}
\begin{flushleft}
    % Global vars
    No global variables plan to be implemented at this time.
\end{flushleft}
\begin{flushleft}
    % Functions
\end{flushleft}
\begin{flushleft}
    % Interrupts and ISRs
    If the charge controller indicates that the battery has fallen below a
    certain voltage (named "low voltage"), the MCU will raise an interrupt and
    execute \texttt{isr\_low\_power()}. This ISR performs housekeeping before
    putting the MCU into a low power state, limiting use of its functions.
\end{flushleft}
\begin{flushleft}
    % Interrupts and ISRs (cont.)
    If the charge controller indicates that the battery has fallen below a
    certain voltage (named "critical voltage"), the MCU will raise an interrupt
    and execute \texttt{isr\_critical\_power()}. This ISR performs further 
    housekeeping before putting the MCU into an extreme low power state,
    limiting all but the features necessary to maintain a connection with AWS.
\end{flushleft}
\begin{flushleft}
    % Interrupts and ISRs (cont.)
    If the MCU, AWS, or any other devices transmit indication of a dangerous
    state or if the MCU, AWS, or any other devices transmit indication of a
    shut down the MCU will raise an interrupt and execute
    \texttt{isr\_shut\_down()}. This ISR immediately shuts down all able
    subsystems, and puts all other subsystems in a fail safe state. This ISR
    fails safe the entire system.
\end{flushleft}
\begin{flushleft}
    % Interrupts and ISRs (cont.)
    If the MCU receives indication of a start up (via a momentary switch), the
    MCU will raise an interrupt and execute \texttt{isr\_start\_up()}. This ISR
    starts up all relevant subsystems and begins the MCU's programming. All
    classes will be initialized to default values. This ISR starts up the
    entire system.
\end{flushleft}
\begin{flushleft}
    % Interrupts and ISRs (cont.)
    If the MCU determines it must reset the system to a default state, the
    MCU will raise an interrupt and execute \texttt{isr\_default\_state()}.
    This ISR returns all relevant subsystems to their default state, almost
    as if the MCU had just executed \texttt{isr\_start\_up()}. All classes will
    be initialized to default values. This ISR resets the entire system.
\end{flushleft}
\begin{flushleft}
    % File organization (main, header files, other source files, etc.)
\end{flushleft}
\begin{flushleft}
    % How the MCU gets data from sensors (ADC)
\end{flushleft}
\begin{flushleft}
    % How MCU sends data to servos
\end{flushleft}
\begin{flushleft}
    % What sort of telemetry we'll have
    No data that would be exclusively considered telemetry will be transmitted 
    between AWS and the MCU (e.g. processor temperature). Instead, all
    "telemetry" values will be handled with interrupts and ISRs/functions
    on-chip, while current settings and sensor readings will be transmitted
    back to AWS.
\end{flushleft}
\begin{flushleft}
    % What kind of development model?
    % Probably agile
    An Agile development model will be used. Code reviews will be performed on
    an as-needed basis by a convening of members of the MCU subsystem and the
    websubsystem teams.
    \begin{figure}[H]
        \label{mcu_agile_uml}
        \caption{UML diagram of the project's development methodology}
        \centering
        \includegraphics[width=0.75\textwidth]{images/mcu_agile_uml.png}
    \end{figure}
\end{flushleft}
\begin{flushleft}
    % IDE and Git
    Texas Instruments Code Composer Studio v12 will be used to program,
    compile (via TI ARM compiler v20), and debug the C++-based project. GitHub
    will be used as a repository for the project, using Git for version
    control.
\end{flushleft}                  % Section 5.1
\subsection{Power Subsystem}
\label{sec:power_subsystem}
The power system is an important part to any electrical device or component that requires any amount of power. Most often, it starts from a power source such as a battery or wall outlet then is converted into energy to operate what is being used. At the minimum, this model is designed to be an independent system, having the capability to operate on its own. The way that can be achieved is through solar power. \par
Solar power has been such a strong growing source of energy and will play a vital role in our system. Power is collected through the solar panels and then regulated through the charge controller to ensure that the battery is receiving the correct amount of charge. The power is regulated through the charge controller so that the battery is not being over charge, which could potentially damage it and reduce the battery life in the future. It is then stored in a battery when the system is not operating and then utilized when needed. From there, voltage needs to be regulated again throught the voltage regulator for the microcontroller and other components that require different amounts of voltage. The block diagram in \autoref{fig:power-block-diagram} shows the flow of power through the system. 

\begin{figure}[H]
    \centering
    \caption{Power subsystem block diagram}
    \includegraphics[width=\textwidth]{images/Power_Subsystem_Block_diagram2.png}
    \label{fig:power-block-diagram}
\end{figure}
The block labeled ``Mechanics'' in \autoref{fig:power-block-diagram} refers to the solenoid valve for controlling water flow as well as the control scheme for actuating the solar panels to maximize power efficiency.
\subsubsection{Solar Panel Control}
To get a greater degree of accuracy the use of stepper motors are used. First, based on the choice of solar panels we get the mass and dimensions of the solar panels, these are .76kg and .336m x .2m respectively. This is important for calculating the torque. We only need to provide two degrees of freedom, one about the short axis (the horizontal axis that bisects the .2m side) and the vertical axis (the axis at the intersection that bisects the two sides of the panel). From the length and mass we can calculate the force per unit length:
\begin{equation}
    \frac{.76 \text{kg}}{.336 \text{m}}=22.62\frac{\text{N}}{\text{m}}
    \label{eqn:short-axis-fpl}
\end{equation}
Now, knowing that torque is $F\times d$ we can formulate the torque for any given angle about this axis in the following:
\begin{equation}
    \int_{0}^{.168} 22.62 \times x \times \sin{\theta} \, dx
    \label{eqn:short-axis-torque}
\end{equation}
\autoref{eqn:short-axis-torque} gives the torque for any given angle. Something you might have noticed is that this is for only one side of the panel. The full torque about the central axis is as follows:
\begin{equation}
    \int_{0}^{.168} 22.62 \times x \times \sin{\theta} \, dx - \int_{0}^{.168} 22.62 \times x \times \sin{180-\theta} \, dx
    \label{eqn:short-axis-torque-total}
\end{equation}
From \autoref{eqn:short-axis-torque-total} we can find the maximum and minimum values of the torque about the axis by finding the crossings of the first derivative with respect to $\theta$ and finding the concavity in the second derivate with respect to $\theta$. Doing so proves the intuition that there is no torque around either axis, this equation evaluates to 0 for all $\theta$.

To save on power \autoref{fig:stepper-config} shows that two motors will work to rotate the panels about the short axis while the center axis is actuated by a single motor and belt system.
\begin{figure}[H]
    \centering
    \caption{Stepper motor configuration}
    \includegraphics[width=0.5\textwidth]{images/stepper-config.png}
    \label{fig:stepper-config}
\end{figure}
The reason we have to use two stepper motors for the short axis rotation is that because as the panels rotate about their central axis any rod or pulley or tensioner system would come out of alignment. Another advantage of this due to the lack of torque in the system is that using a motor controller the two motors can be driven with a single H-bridge with a minimum penalty to power, this will ensure that both panels are always pointed at the same angle in the desired direction.\par
\subsubsection{Voltage Regulator Designs}
As mentioned before, voltage regulators play an important role in regulating voltage throughout the whole model. They make sure that all the components like the microcontroller, sensors, mechanics, etc. are receiving the correct amount of voltage.\par
The CC3220 microcontroller requires 3.3V, the stepper motor requires 5V, and the light source and solenoid require 12V. With the battery running at 12V, the voltage regulators work to supply the correct voltage to each component. To perform and achieve the proper regulation, we created different design schematics for the LM317 (Linear) voltage regulator and the LM2576 (Switching) regulator. Along with that, we created DC/DC designs on WEBENCH as a consideration for specific conditions.\par
\paragraph{Linear Voltage Regulator Designs}
The linear voltage regulator schematic is designed to generate an output voltage of 3.3V for our microcontroller. This schematic is obtained from the LM317 voltage regulator datasheet, Figure 39, and then designed on Multisim with capacitors, resistors, and diodes that are given to us from the software. From the datasheet, there were values that were given and from there we had to find values and then test them. The C_{ADJ} we used 1uF as a constant and then 390Ω was used for the value R. With these inputed values, we achieved an estimated 3.3V and measuring the adjustment current we got 5.2634mA as shown in Figure 41.\par
\begin{figure}[H]
    \centering
    \caption{LM317 Linear Voltage Regulator }
    \includegraphics[width=\textwidth]{images/LM317_Application_schematic.png}
    \label{fig:linear-voltage-regulator}
\end{figure}
\begin{figure}[H]
    \centering
    \caption{LM317 3.3V Linear Voltage Regulator}
    \includegraphics[width=\textwidth]{images/LM317_3.3_schematic.png}
    \label{fig:3.3V-linear-voltage-regulator}
\end{figure}
In Figure 42, we re-designed the ciruit get the regulated output voltage of approximately 5V for the stepper motor. This is done with the same circuit as for the 3.3 output voltage regulation, but replacing the value R with 720Ω and C_{ADJ} still remaining at 1uF. With these values that we found we achieved an output voltage of about 5V and the adjustment current of 5.2617mA.\par
\begin{figure}[H]
    \centering
    \caption{LM317 5V Linear Voltage Regulator}
    \includegraphics[width=\textwidth]{images/LM317_5_schematic.png}
    \label{fig:5V-linear-voltage-regulator}
\end{figure}
\paragraph{}In this testing, we were able to verify that the LM317 Linear voltage regulator can output the proper voltages through simulation. Along with this testing, we also went ahead to see if the voltage regulator was able to output higher voltage levels. For this testing we wanted to see if we could achieve a regulated output voltage of 12V. Using the same circuit for the output voltages 3.3V and 5V, we changed the value of R to increase the output voltage. During this simulation test, we were unable to get the regulated 12V from the output. Instead, the highest that it went up to was about less than 11V. We also noticed that when we past 2kΩ the adjusttment current and the voltage reference started to decrease. Not only that but the output voltage stayed between 10V and 11V, but didn't go higher than 11V.
\begin{figure}[H]
    \centering
    \caption{LM317 10V Linear Voltage Regulator}
    \includegraphics[width=\textwidth]{images/LM317_10_schematic.png}
    \label{fig:10V-linear-voltage-regulator}
\end{figure}
\paragraph{Switching Voltage Regulator Designs}
The same approach was used for the switching voltage regulator to find out how to achieve the correct output voltage. This will be done for the CC3220 microcontroller and sensors that operates at 3.3V and the stepper motor operating at 5V. The Texas Instrument datasheet they had 2 different versions of their schematic done for their fixed version as well as their adjustable version. In the fixed version, we are able to set the output regulated voltages to different levels, such as 3.3, 5, 12, and 15V, leaving the load current fixed to 3A. This is shown in the figure below.\par
\begin{figure}[H]
    \centering
    \caption{LM2576 Fixed Switching Voltage Regulator}
    \includegraphics[width=\textwidth]{images/LM2576_Fixed.png}
    \label{fig:fixed-switching-voltage-regulator}
\end{figure}
For the adjustable version of the LM2576 switching voltage regulator, because it is adjustable, the maximum input voltage that is allowed by this version is 25V. The output regulated voltage can also only output 10V, maximum, with a constant load current of 3A, similar to the fixed version. This is shown in Figure 42. In this version as well we can see that values R1 and R2, on the right, need to be found to find the Vout. This can be done with the equations below as well.\par
\begin{figure}[H]
    \centering
    \caption{LM317 adjustable switching Voltage Regulator}
    \includegraphics[width=\textwidth]{images/LM2576_Adjustable.png}
    \label{fig:adj-switching-voltage-regulator}
\end{figure}

\begin{figure}[H]
    \centering
    \caption{Switching Voltage Regulator Equations}
    \includegraphics[width=\textwidth]{images/LM2576_Adjustable_equations.png}
    \label{fig:switching-voltage-regulator-equations}
\end{figure}
For this adjustable version, there is obviously no fixed value that the output voltage has to be. In this case, let's say the minimum outpute voltage of 10V We would still need to find the value in for 4 and the regulator chosen for. \par
We were also able to find this specific switching voltage regulator on WEBENCH as well, in the figure below. Reading what was on WEBENCH this specific design is able to provide, 86\% efficiency, has a BOM cost of \$9.84, and has a footprint of $957mm^2$. In this circuit design, to achieve the correct regulated output voltage, it is all dependant on the values Rfbt and Rfbb. This gives the ideal resistances that allows the circuit to reach the right voltages. Luckily, were able to find this design on WEBENCH, because it helps have a better understanding for this voltage regulator with the cost and efficiency that it provides. 
\begin{figure}[H]
    \centering
    \caption{LM2576 WEBENCH Design}
    \includegraphics[width=\textwidth]{images/LM2576_WEBENCH.png}
    \label{fig:LM2576 WEBENCH Design}
\end{figure}
\paragraph{WEBENCH Designs}
 WEBENCH is a resource by Texas Instruments that helps design power supplies easily. This resource can be used for DC/DC power systems and AC/DC power systems. It is customizable and helps create various power supply circuits. In this software, we have the freedom to pick and choose our input and output values, and also consider if we want the desired circuit to be balanced, low cost, have high efficiency, or have a small footprint. \par
 For this system, we used input values between 8V and 22V and then output values of 3.3V with max current of 3A. With this we were able to find the LM60430 voltage regulator, an easy-to-use, high efficiency buck converter. This design itself provides an efficiency of 87.5\%, a BOM cost of \$2.39, and footprint of $175mm^2$. This is one of their low cost designs.\par
 \begin{figure}[H]
    \centering
    \caption{LM60430}
    \includegraphics[width=\textwidth]{images/LM60430_WeBench.png}
    \label{fig:WEBENCH Design LM60430}
\end{figure}
The LM34936 voltage regulator schematic was found compatible for the 12V output regulation. Going back and changing the output value to 12V and changing the desired circuit consideration to a balanced circuit we find the LM34936. This is a high-efficient, buck-boost converter that has a wide input range of 4.2V to 30V. The LM34936 provides an efficiency of 97\%, a BOM of \$9.96, but a footprint of $1092mm^2$. \par
\begin{figure}[H]
    \centering
    \caption{LM34936}
    \includegraphics[width=\textwidth]{images/LM34936_WeBench.png}
    \label{fig:WEBENCH Design LM34936}
\end{figure}
Although the LM34936 is bigger, the price for the high efficiency this design provided was have been very good. Another reason why the LM60430 is easy-to-use, is because the size of the schematic.\par
                % Section 5.2
\subsection{Sensing Subsystem}
\begin{figure}[H]
    \caption{Sensing subsystem block diagram}
    \centering
    \includegraphics[width=0.75\textwidth]{images/OpticsBlockDiagram.png}
\end{figure}


The sensing subsystem is an infrared spectrometer that uses two photodiodes as detectors, a diffraction grating as a spectral separator, an array of LEDs to illuminate the target area, and optics to collect, collimate, and focus the beam. In order to determine the component positions and interaction, each component will have to be addressed.
    \begin{itemize}
        \item Ground
        \item LEDs
        \item Fiber in
        \item Fiber out collimated
        \item Reflective Diffraction Grating
        \item Focusing Lens
        \item Carriage driven Photodiodes
    \end{itemize}


    Dirt

That’s right, the first part of the optical subsystem is the soil. Structures in the soil emit trace electromagnetic signals that correspond to the soil matrix, soil temperature, soil moisture, and chemical content of the soil. (Link papers)
In order to ensure sufficient signal to noise ratio, the dirt is going to have to be probed by a bright, broadband source covering the visible and Near Infrared regime. Halogen lamps will do (link papers) but it may be possible to achieve the same effect with LEDs near the surface. 
An additional concern is that irregularities in the soil topography may reflect back light that would otherwise indicate organic components or water. In order to counter this, it is going to be necessary to flatten the dirt with a stamp or roller.


   LEDs

Placeholder until followup with ocean insight.

    Fiber in

placeholder

    Fiber out collimated

The fiber collimator selection hasn’t been finalized yet, but for now I’m assuming a beam of about 2mm diameter will come out of the fiber head collimated toward the grating.

    Grating

    The grating will be a 1200 groove per mm aluminum coating 1000nm blazed reflection grating. 
 
The grating equation governs the angle at which light of different frequencies will reflect. 
The first three orders of diffraction are plotted below for a 1200 groove per mm grating. The polar plot is degrees, ranging from 400nm to 1700nm. Rho represents the order of diffraction.
An angle of incidence of 45 degrees will produce backreflection toward the fiber collimator, however an angle of incidence at 50 degrees will not.

    Focusing Lens

    Carriage driven Photodiodes

    


\begin{figure}[H]
    \caption{Diffraction angle calculation}
    \centering
    \includegraphics[width=0.75\textwidth]{images/DiffractionAngleCalculator.png}
\end{figure}

\begin{figure}[H]
    \caption{Photodiode Signal Filter}
    \centering
    \includegraphics[width=0.75\textwidth]{images/ColimatedBeam.png}
\end{figure}


\begin{figure}[H]
    \caption{Photodiode Signal Filter}
    \centering
    \includegraphics[width=0.75\textwidth]{images/ElectricalSignalFilteringSD1.png}
\end{figure}


The Photodiode works by converting a small portion of the incident light into electrical current across the face of the semiconductor. The diode will have a small current running through the circuit hooked up to an amplifier for boosting the signal to a detectable level. Then another op amp will cut out the electrical noise created by unwanted frequencies generated by the diode and electromagnetic interference.

              % Section 5.3
\subsection{Web Subsystem}
\label{sec:web_subsystem}
The web subsystem will be split into several distinct parts to achieve several goals. First and foremost, the web will communicate over TCP with the MCU to get data, process the data and send commands back to the MCU. Secondly, there will need to be a database in order to log data and be able to support multiple of these garden beds in a scaled solution. Third, there needs to be a user interface that allows the user to set settings and read the data about their garden bed(s). Lastly, the web component of this project will communicate using HTTP requests with a weather service to get upcoming weather such as rain, sunlight, and freeze warnings in order to help the user with maintaining their plant bed.


The block diagram below shows a high level overview of flow of data between different sources. The microcontroller and web component share a two way interface to transmit controller data to the web and commands back to the MCU. Refer to the MCU subsection for more details on the commands. Part of the this component will be a way to process the data received from the weather service and the MCU to serve it to the UI and thus the user. 
\begin{figure}[H]
    \caption{Web component block diagram}
    \centering
    \includegraphics[width=\textwidth]{images/WebBlock.png}
\end{figure}
Section \ref{sec:technology} has information on the technologies in the stack while the following sections will go into more detail about how these choices will be used in order to achieve the goals outlined above. These choices are a React frontend as it is a technology that the team is familiar with and has good support and allows for an event-driven UI as would be useful to serve the polling data collection of the backend. Express will be used as a middleware and routing layer for the user interface to the server backend. This choice was made again to keep the language in JavaScript and the vastness of suppport on the platform. Express also supports Socket.js which will be necessary for communicating over TCP with the microcontroller. MySQL will be used as a relational database as the data is relational. SQL databases also typically serve data-driven applications better over service-driven.
\subsubsection{Server Backend}
As mentioned above, the backend will consist of two parts, a MySQL database as this is a free option and serves the purposes well especially for logging data, as well as an Express backend which will serve all the API endpoints for the user interface as well as a serve to create TCP connections with the microcontroller.
\paragraph{Communication}
This section will cover how we plan on implementing communications with the microcontroller. As covered in \ref{sec:controller_subsystem} these two systems will communicate over TCP. The means of implementing this will be through sockets. This backend will serve as a server to make connections to. Through the use of the Socket.io library, it will be nearly superficial to create this connection and read raw data and process it. Once data is received will be stored in the database for computation.
\paragraph{Database}
The database will be relational. Relating a plant bed and its current state to the logs of data stored within the database as well. The way this will be implemented is there will be a table, \verb|garden_bed| which will store a garden bed and the client socket information. From this point, the ID of the \verb|garden_bed| will be used in a lookup in other tables such as \verb|data_solar| or \verb|data_soil| to be able to present this data to the user in the UI upon request. These data tables will have a timestamp of their creation to be able to lookup the most recent data and retrieve those items and also to create a log and graphs if we so chose. Figure \ref{fig:erd} has some details on the type of data and where it can be found within the database given the design above.
\begin{figure}[H]
    \caption{Entity relationship diagram}
    \centering
    \includegraphics[width=\textwidth]{images/EntityRelation.png}
    \label{fig:erd}
\end{figure}
Another option for the organization of the database is to create a table for each plant bed, which for this project would only be one, that holds all the data and logs regarding to that plant bed. This solution is slightly harder to implement. It is harder to implement because ORMs (Object-Relational Mapping) do not like the dynamic nature of tables. ORMs know the schema of a database table per the name, and when the database becomes highly dynamic in this architecture, the ORM typically requires more work to setup. For more detail on ORMs refer to the technology section of this document.
\paragraph{Data Processing}
All data will be processed on the EC2 instance within the server. Using some heuristic methods and research into peak conditions for different plant types, the server will be able to determine different actions to be made on the plant bed in order to promote growth.
\paragraph{Accounts and Plant Bed Linkages}
Each user account will be linked with potentially multiple plant beds through the database designed above. The backend will validate user logins to endpoints using JWTs in cookies in order to validate each user that logs in.
\subsubsection{User Interface}
The below hex values were selected using a color picking service that guarantees a palette that is useable for users who might be colorblind.
\begin{figure}[H]
    \caption{Color palette}
    \centering
    \includegraphics[width=\textwidth]{images/Color Palette.png}
    \label{fig:color_palette}
\end{figure}
For user experience, catering to colorblindness is important. From left to right the purpose of each color is: component color, emphasis, highlight, background color, error.

To design the user interface we will be using Figma. Figma offers some prototyping and features in order to quickly build a user interface. In conjunctions with React as a framework and CSS, putting everything together should be relatively trivial.                  % Section 5.4
\section{Testing}                               % Section 6
\section{Administrative Content}                % Section 7
\subsection{Milestones}
Our team has decided to utilize the Agile approach for this project. We chose this methodology because it allows our team to focus on sections of the project and aligns well with the semester schedule. By making iterations for this project internally, we are able to track our progress and make status updates. Another benefit of this will be tracking any delays or problems. If we are behind on a section, we have already planned ahead and allowed ourselves room for some variance. We are also using smaller deliverables for the project which gives us more tracking because we have more internal deadlines. We will be using Jira to track our progress and add reports throughout the process. We will also be utilizing Discord as a form of communication with each other. This will be a place where we can discuss any issues or ask quick questions when we are not in person. Below, we have a high-level breakdown written for our project goals.
\subsubsection{Fall}
\begin{itemize}
    \item COMPLETED: Select components for each subsystem
    \begin{itemize}
        \item Document selection reasoning
        \item Order to ensure on-time delivery
    \end{itemize}
    \item Model physical bed
    \begin{itemize}
        \item This item was not completed and will be added to Spring
    \end{itemize}
    \item Build physical bed
    \begin{itemize}
        \item This item was not completed and will be added to Spring
    \end{itemize}
    \item PARTIALLY COMPLETED: Understand how subsystems will integrate:
    \begin{itemize}
        \item Communication protocols (REST, I2C, SPI, DSP, etc)
        \item Power requirements
    \end{itemize}
    \item UI for Web subsystem
    \begin{itemize}
        \item This item was not completed and will be added to Spring.
    \end{itemize}
\end{itemize}
\subsubsection{Spring}
\begin{itemize}
    \item Model physical bed
    \item Build physical bed
    \item UI for Web subsystems
    \item Test subsystems in isolation
    \item Start integrating subsystems
    \item Control scheme for moving solar panels with sun and to provide shade
    \item Web API complete
    \item MCU coding complete
    \item Stretch goals
\end{itemize}

\subsection{Progress}

\subsubsection{Senior Design I}
\begin{figure}[H]
    \caption{Cumulative Flow Diagram from Jira}
    \centering
    \includegraphics[width=\textwidth]{images/Cumulative flow diagram.png}
    \label{fig:cumulativeflow}
\end{figure}
In Figure \ref{fig:cumulativeflow}: purple designates tasks that are marked unfinished in the the backlog and current sprint, blue represents in progress tasks, and green represents finished tasks.

Throughout Senior Design I we have been gathering research and have started laying out the design of our garden bed and have completed the majority of our part selection. The Figure in \ref{fig:cumulativeflow} may be a little misleading at this point because we have not refined our backlog to fully encapsulate meaningful tasks instead breaking it down into larger subsystem requirement-esque tasks.
%Needs updating for SD2
            % Section 7.1
\subsection{Budget}
The budget was created from the individual subsystem BOM. The BOMs for each subsystem can be found in the respective part selection section. The budget is broken down into the 4 subsystems as well as on miscellaneous section for the physical bed. 
\begin{table}[H]
    \centering
    \begin{tabularx}{.8\textwidth}
        {
            | >{\raggedright\arraybackslash}X
            | >{\raggedright\arraybackslash}X
            | >{\raggedleft\arraybackslash}X
            |
        }
        \caption{Breakdown of budget by subsystem}\\
        \hline
        Subsystem & Estimated Cost & Comment \\
        \hline
        MCU & \$60 & The MCU, wiring harness \\         % Changed 2022-09-29 by Brendan
        \hline
        Power & \$200 & Solar panels, batteries, control system \\
        \hline
        Sensing & \$700 & Components for sensing, optical sensors \\ % Changed 2022-12-6 by Scott
        \hline
        Web & \$30 & Web service pricing \\             % Changed 2022-09-29 by Brendan
        \hline
        Non-Subsystem & \$100 & The plant bed, soil, water, fittings, etc \\
        \hline
        Total & \multicolumn{2}{|c|}{\$1090}\\           % Changed 2022-12-6 by Scott
        \hline
    \end{tabularx}
\end{table}

\subsection{Total Cost}
The budget cost table and the real cost data table show some differences in terms of estimated costs versus actual costs. The budget cost table estimated a total cost of \textdollar1090, but the real cost data shows a total cost of \textdollar1543.26, which is over 40\% more than the estimated cost. Looking at the breakdown by subsystem, the biggest discrepancies appear to be in the sensing and control categories, which had estimated costs of \textdollar700 and \textdollar60 respectively, but actual costs of \textdollar680.22 and \textdollar332.07. On the other hand, the power and non-subsystem categories were closer to their estimated costs. Despite the differences, the real cost data table provides a comprehensive breakdown of the actual costs incurred, allowing for a more accurate assessment of the project's expenses.
\begin{table}[H]
    \centering
    \begin{tabularx}{.8\textwidth}
        {
            | >{\raggedright\arraybackslash}X
            | >{\raggedright\arraybackslash}X
            | >{\raggedleft\arraybackslash}X
            |
        }
        \caption{Total cost itemized}\\
        \hline
        Description & Category & Cost (USD) \\
        \hline
        IR photodiode & Sensing & 50.21 \\
        Linear Rail Actuator & Sensing & 30.76 \\
        Tungsten Lamp & Sensing & 16.4 \\
        DRV8833 PCB & Sensing & 15.54 \\
        JLCPCB Sensor PCB v1.2 and stencil & Sensing & 61.96 \\
        Digikey order & Misc & 90.01 \\
        Cylindrical Lens & Sensing & 67 \\
        Fiber Patch Cable & Sensing & 87.24 \\
        Fiber Collimator 1 & Sensing & 177.45 \\
        Motor Drivers & Sensing & 22.22 \\
        Dev board and IC & Control & 131.06 \\
        Solenoid, Antenna, CC3200 & Control & 54.06 \\
        AD8656 & Sensing & 52.83 \\
        Main and Sensing Containers from Harbor Freight & Misc & 26.6 \\
        12VLiFePo4 Battery & Power & 74.99 \\
        Solar Panel & Power & 52.98 \\
        BQ25713RSNR & Power & 9.38 \\
        LM317KCS & Power & 1.67 \\
        LM2576S & Power & 10.34 \\
        Remaining BOM Order Digikey & Misc & 126.23 \\
        JLCPCB MCU PCB v1.1 & Control & 94.86 \\
        JLCPCB Sensor PCB v2.1 & Sensing & 47.45 \\
        Power PCB & Power & 105.33 \\
        Digikey order 3/10/23 & Misc & 22.78 \\
        TI ADS7142 eval board & Sensing & 38.63 \\
        JLCPCB MCU v2.0 & Control & 52.09 \\
        ACE Hardware mounting stuff & Misc & 10.66 \\
        ACE bulb trip 1 & Sensing & 3.99 \\
        ACE bulb trip 2 & Sensing & 8.54 \\
        \hline
        Total balance & & 1543.26 \\
        \hline
    \end{tabularx}
\end{table}

\begin{table}[H]
    \centering
    \begin{tabularx}{.8\textwidth}
        {
            | >{\raggedright\arraybackslash}X
            | >{\raggedright\arraybackslash}X
            | >{\raggedleft\arraybackslash}X
            |
        }
        \caption{Total cost by area} \\
    \hline
    \textbf{Category} & \textbf{Cost} & \textbf{\% of Total} \\
    \hline
    Control & 332.07 & 21.51\% \\
    Power & 254.69 & 16.50\% \\
    Sensing & 680.22 & 44.10\% \\
    Misc & 276.28 & 17.89\% \\
    \hline
    \textbf{Total} & \textbf{1543.26} & \textbf{100\%} \\
    \hline
    \end{tabularx}
\end{table}

This table shows the cost breakdown of the project by category, as well as the percentage of the total cost each category represents. Control accounts for 21.51\% of the total cost, Power accounts for 16.50\%, Sensing accounts for 44.10\%, and Misc accounts for 17.89\%. The total cost of the project is 1543.26.               % Section 7.2
\end{document}