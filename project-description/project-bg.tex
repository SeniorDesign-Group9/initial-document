\subsection{Project Background}
Home gardening is a valuable hobby that helps people get outdoors, create something beautiful, and earn tangible rewards. Unfortunately, plants are sensitive organisms that require constant attention. Plant life also involves complex relationships between the organisms and their environment. These two problems can discourage potential gardeners. This team proposes to build a system that addresses these issues through mechanical automation and user notification.

In 2022 there is great potential for realizing these advantages. Remote sensing, wireless communication, API integration, and closed-system power and water control are both available and economical. Many “DIY smart-gardens” and “Garduino” projects have been published to the internet. In the agriculture industry, there are commercially available technologies with high performance systems for water distribution, network communication, and remote sensing. This project is intended to advance the field by producing a system that can maintain a suitable environment for plant growth by autonomously sensing and modifying the conditions of the garden bed. In addition, it will feature a notification system that allows the plants to “speak” by prompting the user to make accommodations when bad weather is projected, such as a cold snap.

Especially noteworthy is that this project features an on-the-rise technology in the form of a Near Infrared Spectrometer. Smart agricultural systems need to determine several variables, including moisture level, nutrient content, and acidity. There is a wide variety of techniques for sensing soil moisture, but by far the cheapest and most used is electrical conductivity. This involves pressing electrical nodes into the ground and up against the wet soil matrix, which inevitably leads to corrosion. DIY and commercial systems require other discrete sensors and even chemical analysis to characterize the state of the soil. Near Infrared Spectroscopy is an alternative method of soil sensing that offers many advantages over these traditional technologies. It works by stimulating a response from weak molecular bonds in the soil, isolating the frequencies of that response, and comparing the signal strengths to that of a known sample. In addition to being a noninvasive, corrosion-immune source of information about soil moisture, the NIR Spectrometer gathers data about the chemical contents of the soil. This means that the same scan will detect the presence and quantity of soil nutrients and water acidity as well. A high speed, high precision spectrometer costs between \textdollar5,000 - \textdollar10,000, but this application requires only rudimentary sensing capabilities. The team was able to accomplish building the spectrometer for less than \textdollar800 which is a vast improvement to the commercial options.

\subsubsection{Motivation}
The idea came from seeing the ``Garduino'' style projects all over hobbyist forums and websites but the idea really took hold in that each member of the team saw an opportunity to explore a new facet of engineering they held an interest in. This project provided the team an opportunity to apply our knowledge on power systems and delivery, controls, digital signal processing, and optical sensing. These are all areas that the team wanted to demonstrate a high level of understanding in and grow at the synthesis level.